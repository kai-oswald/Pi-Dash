\section{Anhang}
\subsection{Skripte}
\label{sec:Skripte-1} 

\subsubsection{UDP Skript für den Raspberry PI:}$\;$ \\  
\label{sec:UDPAnhang}
Das Skript, welches alle UDP Pakete der Buttons entgegennimmt, geschrieben in der Sprache Python:
\lstinputlisting[language=Python]{textteile/udp.py}
\newpage

\subsubsection{TCP Skript für den Raspberry PI:}$\;$ \\  
\label{sec:TCPAnhang}
Das Skript, welches alle TCP Pakete der Buttons entgegennimmt, geschrieben in der Sprache Python:
\lstinputlisting[language=Python]{textteile/tcp.py}
\newpage

\subsubsection{ARP Skript für den Raspberry PI:}$\;$ \\  
\label{sec:ARPAnhang}
Das Skript, welches alle ARP Pakete der Buttons entgegennimmt, geschrieben in der Sprache Python:
\lstinputlisting[language=Python]{textteile/amazon.py}
\newpage

\subsubsection{Service Skript für den Raspberry PI:}$\;$ \\
\label{sec:ServiceAnhang}
Das Skript, welches die unterschiedlichen Netzwerkskripte (vgl. \ref{sec:UDPAnhang}, \ref{sec:TCPAnhang} und \ref{sec:ARPAnhang}):
\lstinputlisting[language=Python]{textteile/service.py}
\newpage

\subsubsection{MySQL Skript}
\label{sec:MySQLSkript}
Die Befehle um die MySQL Datenbank für das Projekt einzurichten: 
\lstinputlisting[language=Python]{textteile/mysqlscript.txt}
\newpage 

\subsection{Konfigurationsdateien}  
\label{sec:Konfigurationsdateien-1} 

\subsubsection{Nginx Konfiguration}
\label{sec:NginxKonfiguration}
Die Konfigurationsdatei des Nginx Webservers: 
\lstinputlisting[language=Python]{textteile/nginxconfig.txt}
\newpage 

\subsubsection{Hostapd Konfiguration}
\label{sec:HostapdSkript}
Die Konfigurationsdatei des Diensts hostapd:
\lstinputlisting[language=Python]{textteile/hostapdconf.txt}
\newpage 

\subsubsection{Interfaces Konfiguration}
\label{sec:InterfacesConfig}
Die Konfigurationsdatei ``interfaces'':
\lstinputlisting[language=Python]{textteile/interfaces.txt}
\newpage 

\subsection{Programmcode}  
\label{sec:Programmcode-1} 

\subsubsection{Code vom Pretzelboard}
\label{sec:CodePretzelboard}
Der Code, der auf das Pretzelboard gespielt wurde:
\lstinputlisting[language=Python]{textteile/Pretzel.ino}
\newpage 

\subsubsection{Code vom ESP8266}
\label{sec:CodeESP}
Der Code, der auf den ESP8266 gespielt wurde:
\lstinputlisting[language=Python]{textteile/NodeMCU.ino}
\newpage 
