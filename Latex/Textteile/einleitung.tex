Aufgrund der technologischen Entwicklung in den letzten Jahren sind immer kleinere Geräte möglich, die immer mehr Funktionen übernehmen können. Diese Geräte werden in immer mehr Bereichen des täglichen Lebens eingesetzt und ermöglichen so neue Anwendungsgebiete, die sich nicht mehr nur auf den heimischen Computer beschränken. Eine dieser Entwicklungen ist der sogenannte ``Amazon Dash Button'', der von Amazon im Jahr 2015 veröffentlicht und auf den Markt gebracht wurde. 
Dieser kleine Button wurde entwickelt, um den Bestellprozess beim Internethändler zu vereinfachen. Über eine WLAN Verbindung und eine einmalige kurze Konfiguration wird der Kaufprozess für ein zuvor definiertes Produkt angestoßen. Dazu muss nur der Button betätigt werden und die Bestellung wird aufgegeben. (vgl. \cite{Amazon.})

Im Rahmen dieser Studienarbeit werden diese Prozesse untersucht und eine offenere Lösung entwickelt. Bei der ersten Betrachtung des Produkts fällt nämlich auf, dass es für den Kunden einige Nachteile gibt. Zum einen ist er an den Internethändler Amazon gebunden, da dieser nur die Bestellung über sein Angebot ermöglicht. Weiterhin ist es nicht möglich bei der Betätigung des Buttons den aktuellen Preis des Produkts zu sehen. Das bedeutet, dass der Kunde das übliche Produkt bestellt und den genauen Preis nicht weiß. Diese zwei Nachteile sollen durch eine offene und durch den Nutzer konfigurierbare Lösung behoben werden. Auf den folgenden Seiten wird auf die Theorie und die Entwicklung der Lösung detailliert eingegangen und zudem der Amazon Dash Button genauer untersucht.