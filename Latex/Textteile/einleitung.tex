In den letzten Jahren konnten aufgrund technologischer Entwicklungen immer mehr Geräte entwickelt werden, die in den verschiedensten Bereichen des täglichen Lebens eingesetzt werden. Dabei sind die Geräte meist leistungsfähiger oder kleiner geworden, sodass sie in unterschiedlichsten Bereichen eingesetzt werden können. Ein Begriff der in den letzten Jahren für dieses Phänomen immer häufiger genutzt wurde, ist das ``\ac{IoT}'' zu deutsch ``Das Internet der Dinge''. Mit dem \ac{IoT} wird beschrieben, dass immer mehr Geräte mit dem Internet verbunden sind und Daten austauschen. Im Rahmen dieser Entwicklung sind verschiedenste Nutzungsszenarien umgesetzt worden, die das tägliche Leben erleichtern oder automatisieren sollen. Diese Studienarbeit wird sich im folgenden mit einem speziellen Szenario befassen, welches unter anderem durch den Onlinehändler Amazon umgesetzt wurde.

Das Szenario umfasst sogenannte ``Amazon Dash Buttons''. Diese Buttons sind kleine Geräte, welche an unterschiedlichen Stellen in der Wohnung des Kundens angebracht werden können und über einen Druckknopf verfügen. Nach einer Konfiguration durch den Benutzer kann mithilfe eines Drucks auf den Sensor ein zuvor ausgewähltes Produkt bestellt werden. Ein Beispiel wäre das nachbestellen von Waschpulver. So könnte der Button an der Waschmaschine befestigt sein und bei einem geringen Vorrat an Waschpulver wird der Button betätigt und das ausgewählte Waschpulver innerhalb der nächsten Tage geliefert. So entfällt für den Nutzer der händische Bestellvorgang, da dieser automatisch durch den Button und Amazon durchgeführt wird. Dieses Beispiel lässt sich natürlich auf verschiedene andere Möglichkeiten übertragen, so könnten auch Dinge, wie Kaffee. Zahnpasta, Tierfutter, Getränke oder ähnliches nachbestellt werden. Dem Nutzer wird eine Vereinfachung des Kaufprozesses versprochen und der Händler bindet den Kunden stärker an sich, da der Button direkt bei Amazon bestellt. 
Die Studienarbeit setzt sich zum Ziel eine Untersuchung des Amazonproduktes durchzuführen und eine offene Alternative zu entwickeln. 