Mithilfe der Technologien und Softwarelösungen, die in Kapitel \ref{sec:Theorie-1} erläutert worden sind, und der Hardware, die in Kapitel \ref{sec:Beschreibung der Hardware-1} vorgestellt wurde, konnten die verschiedenen Komponenten, die bereits in der Aufgabenstellung (vgl. Kapitel \ref{sec:Aufgabenstellung-1}) genannt worden sind, umgesetzt werden. 
Im folgenden soll ein Überblick über die Zusammenarbeit der einzelnen Komponenten gegeben werden, bevor in den folgenden Kapiteln auf die genaue Entwicklung der einzelnen Komponenten eingegangen wird. 

Für den Nutzer gibt es ein Webfrontend, welches mithilfe eines Nginx Webservers (vgl. Kapitel {sec:WebserverNginx-1}) realisiert wird. Dieses Webfrontend greift auf ein \ac{REST} \ac{API}-Backend zurück, welches in \ac{PHP} geschrieben wurde und ebenfalls über den Nginx realisiert wird. Mithilfe dieses Backends werden die verschiedenen \ac{API} Routen realisiert, die dann den Zugriff auf die Datenbank organisieren. Das Backend besteht ebenfalls aus einigen Skripten, die in Python\ref{src:Python-1}) geschrieben sind. Diese Skripte organisieren die Kommunikationsendpunkte für die Buttons, indem sie auf die entsprechende Kommunikationsports für die Datenpakete lauschen und die Daten dann verarbeiten. 

Neben der Kommunikation wird ebenfalls ein Teil der \ac{REST} \ac{API} in Python realisiert. Dieser Teil ermöglicht sowohl den Status der Skripte, die die Kommunikation organisieren, abzufragen als auch die Skripte neu zu starten, sollte ein Skript nicht gestartet sein oder ein Fehler aufgetreten sein. Diese Daten werden ebenfalls für das Webfrontend verwendet. 
Die Buttons, als weitere Komponente, werden durch verschiedene Hardwareelemente realisiert. Mithilfe entsprechender Programme, die auf die Controller geladen werden, stellen sie eine Verbindung zu den definierten Kommunikationspunkten her. Über dieses Kommunikationsprotokoll werden dann, sobald der Button betätigt wurde, die entsprechenden Daten gesendet und dann im Backend verarbeitet. 
Zur Verarbeitung der Daten gehört auch das Einfügen in die entsprechenden Tabellen der SQL Datenbank, die auf einem MySQL Datenbankserver läuft, der für die Verwaltung der Daten zuständig ist. 
% Vielleicht noch ein passendes Modell einfügen
