Mithilfe der Skriptsprache Python werden die verschiedenen Kommunikationsendpunkte in Form von Sockets realisiert. Um eine bessere Übersichtlichkeit zu gewährleisten und eine zentrale Verwaltung zu haben, gibt es ein Verwaltungsskript. Dieses Skript startet die anderen Skripte, die sich um jeweils einen Kommunikationsprotokoll kümmern. So gibt es ein Skript für die Kommunikation über \ac{UDP}, eins für \ac{TCP} und eins für \ac{ARP}, welches für die Amazon Dash Buttons genutzt wird. Aus diesen Gründen gibt es insgesamt vier Python Skripte, die einen wesentlichen Teil des Backends ausmachen.

\paragraph{Entwicklung des Verwaltungsskripts}$\;$ \\  
\label{sec:Entwicklung des Verwaltungsskripts-1} 
Das Verwaltungsskript dienst, wie bereits erwähnt, als zentrales Skript, welches als einziges Skript auch gestartet werden muss. Über dieses Skript werden dann alle anderen notwendigen Skripte gestartet, die dann dafür sorgen, dass die Kommunikation ermöglicht wird.
Neben dieser Funktionalität wird durch das Verwaltungsskript auch eine \ac{REST} \ac{API} realisiert. Diese wird mithilfe des Frameworks Flask (vgl. \cite{.s}) umgesetzt. Diese \ac{API} wird dazu genutzt, um einige Funktionalitäten bereitzustellen, die im Frontend benötigt werden. Als Beispiel wäre der aktuelle Status der Empfängerskripte (vgl. Kapitel \ref{sec:Entwicklung des UDP Skripts-1} und Kapitel \ref{sec:Entwicklung des TCP Skripts-1}) zu nennen. 

Die genannte \ac{REST} \ac{API} kann dann im Verwaltungsskript auf andere Methoden zugegriffen werden, die dann weitere Funktionen umfassen. Die Rückgabewerte dieser Funktionen werden dann im \ac{JSON} Format an das Frontend der Anwendung weitergegeben und können dann dort weiter verarbeitet werden. 

\paragraph{Entwicklung des UDP Skripts}$\;$ \\  
\label{sec:Entwicklung des UDP Skripts-1} 
Da die Übertragung der Datenpakete über das \ac{UDP} Protokoll ermöglicht werden soll, muss ein entsprechender Empfänger auf dem Raspberry PI vorhanden sein. Dieser Empfänger wird mithilfe eines Skriptes in Python realisiert.

Dieses Skript nutzt die Library ``Socket'' (vgl. \cite{.20.02.2017}), welches es ermöglicht ein Socket zu erstellen. Dieses Socket wird an eine \ac{IP} Adresse und einen Port gebunden und wird anschließend für alle eingehenden \ac{UDP} Pakete genutzt. In einer Endlosschleife, welche manuell unterbrochen werden kann, wird auf eingehende Pakete gewartet. Die Endlosschleife wird benötigt, da ein Button zu jedem Zeitpunkt betätigt werden kann und somit dauerhaft auf ankommende Pakete geachtet werden muss. 

Bei Eingang eines entsprechenden Pakets wird ein entsprechendes Request an die \ac{REST} \ac{API} geschickt. Da als Übertragungsart allerdings das UDP Protokoll genutzt wird, kann dem Button keine Rückmeldung gegeben werden, ob der Eintrag in die Datenbank über die \ac{REST} \ac{API} erfolgreich war. Zudem kann auch kein allgemeines Feedback zurückgegeben werden. Auch bei anderen Fehlern kann dem Button keine Rückmeldung gegeben werden. 

Nach dem erfolgreichen Absenden der Anfrage an die \ac{REST} \ac{API} befindet sich das Skript für den \ac{UDP} Empfänger weiterhin in der Endlosschleife und wartet auf das nächste Datenpaket. 
Das Skript ist auch im Anhang unter \ref{sec:UDPAnhang} zu finden. 

\paragraph{Entwicklung des TCP Skripts}$\;$ \\  
\label{sec:Entwicklung des TCP Skripts-1} 
Für alle Datenpakete, die über das Protokoll \ac{TCP} empfangen werden, muss ebenfalls ein Skript geschrieben werden, welches diese Datenpakete verarbeitet. Dabei wird genauso vorgegangen, wie beim Skript für UDP. Der einzige Unterschied ist nach dem Absenden der Anfrage an die \ac{REST} \ac{API}. Das Skript für \ac{TCP} Datenpakete wartet nach dem Absenden der Anfrage auf die Bestätigung und schickt eine entsprechende Antwort zurück an den Button. Dieser verarbeitet ebenfalls die Antwort und kann mithilfe einer Lampe dem Nutzer ein visuelles Feedback geben. 
Das entsprechende Skript ist im Anhang unter \ref{sec:TCPAnhang} zu finden.

\paragraph{Entwicklung des ARP Skripts}$\;$ \\  
\label{sec:Entwicklung des ARP Skripts-1} 
Da das Mitlesen der \ac{IP} Datenpakete des Amazon Dash Buttons nicht möglich war, wurde das mitschneiden der \ac{ARP} Datenpakete notwendig. 
Im Vergleich zu den Skripten, die in den Kapiteln \nameref{sec:Entwicklung des UDP Skripts-1} und \nameref{sec:Entwicklung des TCP Skripts-1} beschrieben sind, ist dieses Skript etwas anders aufgebaut. Es wird zwar ebenfalls die Library ``Socket'' genutzt, allerdings ist die Konfiguration des Sockets anders, damit \ac{ARP} Datenpakete gelesen werden können. 


Das Lesen dieser Datenpakete beschränkt sich allerdings auf das Erkennen der \ac{IP} Adresse, die das Paket abschickt bzw. empfängt. Nur diese Informationen werden benötigt, da die Funktionalität des Skript darauf beruht, dass jeder zweite \ac{ARP} Request eine entsprechende Anfrage an die \ac{REST} \ac{API} abschickt. Das nur jeder zweite Request eine entsprechende Anfrage abschickt, ist damit begründet, dass bei einem einzelnen Druck auf den Amazon Dash Button ein \ac{ARP} Paket von Button zum Router geschickt wird und ein Paket vom Router zum Button. Das bedeutet, dass für jeden Druck zwei Requests abgeschickt werden. Daher löst nur jedes zweite Paket eine Anfrage aus, die dann die Anzahl eines Produktes in der Datenbank erhöht. 
Das entsprechende Skript ist im Anhang unter \ref{sec:ARPAnhang} zu finden.