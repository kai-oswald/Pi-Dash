% -> eher auf die Aufgabenstellung eingehen 
Nach Abschluss des Projektes gilt es das Ergebnis des Projektes zusammenzufassen und zu bewerten. Im Rahmen der vorherigen Kapitel wurden sowohl auf den theoretischen Teil des Projekts als auch die praktische Durchführung eingegangen. Abschließend gilt es auf die zuvor in der Aufgabenstellung (vgl. Kapitel \ref{sec:Aufgabenstellung-1}) definierten Teilaufgaben einzugehen und zu bewerten, gefolgt von einem abschließenden Ergebnis. 

Die Aufgaben waren in folgende Teilaufgaben eingeteilt: ``Entwicklung von Buttons'', ``Entwicklung eines Frontends zur Nutzerinteraktion'' und ``Entwicklung eines Backends zur Verarbeitung von Frontendeingaben und der Kommunikation mit Buttons'' (vgl. Kapitel \ref{sec:Aufgabenstellung-1}). Im Folgenden sollen die Ergebnisse jeder einzelnen Kategorie zusammengefasst werden:

\subsection{Entwicklung von Buttons}
\label{sec:ErgebnisButtons}
Im Rahmen dieser Kategorie galt es eigene Buttons zu entwickeln, die es ermöglichen eine Machbarkeitsüberprüfung durchzuführen. Es war im Rahmen des Projektes möglich, dass nicht nur ein einzelner Button nachgebaut werden konnte, sondern zwei unterschiedliche Buttons mit unterschiedlichen Übertragungsprotrokollen (\ac{TCP} und \ac{UDP}) erstellt werden konnten. Dadurch war es möglich, dass die Machbarkeit eines solchen Buttons sogar mit unterschiedlichen Möglichkeiten nachgewiesen werden konnte (vgl. Kapitel \ref{sec:Entwicklung der Buttons-1}).

Zudem konnte ebenfalls überprüft werden, ob der Amazon Dash Button in diese Lösung integriert werden könnte. Dies konnte ebenfalls erfolgreich gezeigt werden, auch wenn es eine andere Methodik erforderte als bei den selbstentwickelten Buttons.

\subsection{Entwicklung eines Frontends zur Nutzerinteraktion}
\label{sec:ErgebnisFrontend}
Es galt ebenfalls ein Frontend zu gestalten, welches dem Nutzer eine Einkaufsliste darstellen kann. Dies ist im Rahmen des Projektes ebenfalls gelungen und es ist zu bemerken, dass in diesem Rahmen auch darauf geachtet wurde, dass beispielsweise eine \ac{REST} \ac{API} implementiert wurde, die für weitere Entwickler eine Schnittstelle bereitstellt und somit eine möglichst offene Lösung darstellt, deren Frontend weiterhin erweitert werden kann. Zudem war es möglich noch einige weitere, zusätzliche Funktionen einzufügen, beispielsweise eine Steuerung für das Python Backend (vgl. Kapitel \ref{sec:Aufbau und Entwicklung der Python Skripte-1}). 

Allerdings ist ebenfalls zu erwähnen, dass darauf geachtet wurde, dass die Oberfläche möglichst simpel gehalten wurde, um die Nutzung der Webseite möglichst einfach zu gestalten. 
Insgesamt wurde auch bei dieser Teilaufgabe gezeigt, dass es möglich war, dass die Aufgabenstellung erfüllt werden konnte und eine erfolgreiche Umsetzung möglich war (vgl. Kapitel \ref{sec:Entwicklung der Frontends-1}).

\newpage

\subsection{Entwicklung eines Backends als Schnittstelle für das Frontend und die Buttons}
\label{sec:ErgebnisBackend}
Für den Umfang des Projektes war auch die Implementierung eines Backends von großer Bedeutung, da dieses alle relevanten Eingaben verarbeiten muss. Bei der Umsetzung galt es darauf zu achten, dass sowohl für die Webseite ein entsprechendes Backend gestaltet als auch für die Kommunikation mit den Button ein entsprechendes Backend zur Verfügung gestellt werden musste. 

Aufgrund der Tatsache, dass diese beiden Teile getrennt wurden, war es möglich, dass auf die spezifischen Merkmale beider Teile einzugehen. So war es möglich, dass für die Kommunikation mit den Buttons auf die Skriptsprache Python zurückgegriffen werden konnte, während für die Webseite als Backendsprache \ac{PHP} genutzt werden konnte. Diese Entscheidung ermöglichte eine erfolgreiche Umsetzung des Backends, sodass als Ergebniss ein funktionsfähiges Backend entwickelt werden konnte. Diese Backend ist in der Lage alle erforderlichen Kommunikationsschnittstellen bereitzustellen. Zudem wurde es möglichst modular entwickelt, sodass eine Erweiterung, beispielsweise durch weitere Pythonskripte, möglich ist (vgl. Kapitel \ref{sec:Entwicklung und Einrichtung des Backends-1}). 

\subsection{Abschließendes Ergebnis}
\label{sec:ErgebnisAbschluss}
Nach der Darstellung der drei Teilaufgaben gilt es das gesamte Ergebnis zu betrachten. Aufgrund des Erfolgs in allen drei Teilaufgaben war es möglich, dass das Projekt erfolgreich abgeschlossen werden konnte. Durch die erfolgreiche Zusammenarbeit der drei Teilaufgaben war es möglich zu zeigen, dass ein ähnliches Prinzip, wie das der Amazon Dash Buttons, auch ohne eine Bestellpflicht bei einem Anbieter, umzusetzen ist. Als Ergebnis hat der Nutzer eine Einkaufsliste, die er für verschiedene Zwecke nutzen kann und die nicht an die Preisgestaltung eines Anbieters gebunden ist. 

Zudem ist es möglich den Zeitpunkt der Bestellung selbst zu definieren, da die einzelnen Elemente der Einkaufsliste zu unterschiedlichen Zeitpunkten auch wieder entfernt werden können. 
Aufgrund der unterschiedlichen Konfiguration der Buttons war es zudem möglich aufzuzeigen, dass auch unterschiedliche Kommunikationsprotrokolle genutzt werden können, sodass ein möglicher Nutzer diese Ausarbeitung auf unterschiedliche Arten selbst umsetzen kann. Als abschließendes Ergebnis ist zu erkennen, dass die Aufgabenstellung des Projektes erfüllt wurde und ein gutes Ergebnis erzielt werden konnte. 
