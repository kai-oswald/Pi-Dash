Die Einrichtung des Raspberry PIs besteht aus mehreren Schritten an dessen Ende die Verwendung des Raspberrys als zentraler Server steht. Die verschiedenen Schritte werden im folgenden erklärt: 
\paragraph{Einrichtung des Nginx}  
\label{sec:Einrichtung des Nginx-1} 

\paragraph{Einrichtung des SQL Datenbankservers}$\;$ \\
\label{sec:Einrichtung des SQL Datenbankservers-1} 

\paragraph{Einrichtung des WLAN Access Points}$\;$ \\ 
\label{sec:Einrichtung des WLAN Access Points-1} 

% erklären warum beides (UDP/TCP)
\paragraph{Einrichtung von Python:}$\;$ \\
\label{sec:Python Skripte-1} 
Im Rahmen der Einrichtung des Raspberry Pi's ist auch die Einrichtung von Python durchzuführen. Für eine erfolgreiche Einrichtung müssen die entsprechenden Softwarepakete installiert werden.  Neben der grundlegenden Instalaltion von Python galt es auch die entsprechenden Skripte zu entwickeln. Dies wird allerdings genauer im Kapitel \ref{sec:Aufbau und Entwicklung der Python Skripte-1} betrachtet. 
%was ist mit PIP?
