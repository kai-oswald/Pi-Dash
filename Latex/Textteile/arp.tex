Das \ac{ARP} Protrokoll ist ein Protrokoll, dass auf der zweiten Ebene (Sicherungsebene) des OSI Modells wiederzufinden ist. Mithilfe von \ac{ARP} werden in einem Netzwerk die vergebenen \ac{IP} Adressen in MAC Adressen beziehungsweise Hardwareadressen ungewandelt. Dieser Prozess ist notwendig, da innerhalb eines lokalen Netzwerkes die Netzwerkpakete über diese MAC Adressen an die entsprechenden Empfänger geleitet werden. Da zuvor die Kommunikation über \ac{IP} lief, muss diese Umwandlung durchgeführt werden. 
Diese Unwandlung wird mithilfe von \ac{ARP} realisiert. Sobald ein Gerät einem Netzwerk beitritt, wird ein \ac{ARP} Request an die Adresse ``FF-FF-FF-FF-FF-FF'' geschickt. Diese Adresse ist ein sogenannter Broadcast, der diesen Request an alle Geräte im Netzwerk schickt. Dieser Request wird vom Router so verarbeitet, indem er der, im Request mitgesendeten, MAC Adresse des Geräts eine \ac{IP} Adresse zuordnet, die für das IP Protrokoll verwendet werden kann. Nach diesem erfolgreichen Mapping einer \ac{MAC} Adresse im Netzwerk auf eine \ac{IP} Adresse, kann die weitere Kommunikation beispielsweise über das Internet Protrokoll oder andere Protrokolle genutzt werden. Aufgrund des hinterlegten Mappings weiß der Router nun, zu welcher \ac{MAC} Adresse er die entsprechenden Pakete mit einer bestimmten \ac{IP} Adresse weiterleiten muss. (vgl. \cite{.r}\cite{.s})