Das Slim-Framework bietet eine Option Gruppen für Routen anzulegen. Dies ist perfekt, um APIs zu definieren. Dadurch kann die API anhand der Route /api aufgerufen werden. Folgend ist der Quellcode für die API der Produkte. Um den Quellcode etwas zu erklären, soll die zweite Zeile betrachtet werden. Diese beschreibt die Route /api/products/ bei einer GET-Anforderung. Hier wird auch deutlich, wie mächtig ein \ac{ORM} sein kann. Es werden alle Produkte im JSON-Format als Antwort zurückgegeben.
\lstset{language=PHP} 
\begin{lstlisting}[frame=single]
$app->group("/api", function() use ($app) {  

$app->get("/products/", function($req, $res, $args) {
return $res->withJson(Product::all());
});

$app->get("/products/{id}", function($req, $res, $args){
$product = \Product::find($args["id"]);
if(sizeof($product) == 0) {
return $res->withJson(CreateErrorMessage($args["id"]), 404);
}
return $res->withJson($product);
});

$app->post("/products/", function($req, $res, $args) {
$request = $req->getParsedBody();
try {
$product = new Product;
$product->name = $request["name"];
$product->price = $request["price"];
$product->save();
return $res->withJson($product);
}
catch(Exception $e) {
return $res->withJson($e, 400);
}
});

$app->put("/products/{id}", function($req, $res, $args) {
$request = $req->getParsedBody();
try {
$product = \Product::find($args["id"]);
$product->name = $request["name"];
$product->price = $request["price"];
$product->save();
return $res->withJson($product);
}
catch(Exception $e) {
return $res->withJson($e, 400);
}
});

$app->delete("/products/{id}", function($req, $res, $args) {
try {
$product = \Product::find($args["id"]);
$product->delete();
return $res->withJson($args[id]);
}
catch(Exception $e) {
return $res->withJson($e, 400);
}
});
\end{lstlisting}