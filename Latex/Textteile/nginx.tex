Nginx besitzt einen Marktanteil von ca. 20 Prozent aller Webserver und ist somit auf Rang 2 der Webserver für aktive Webseiten hinter Apache und vor Microsoft und Google\cite{.k}. In dieser Studienarbeit haben wir uns für nginx entschieden, da ein hoher Wert auf eine flexible Konfiguration und schnelle Serverantworten gelegt wird.

\subsubsection{Allgemein}
\label{sec:NginxAllgemein}

nginx ist ein HTTP und Reverse-Proxy Server, ein Mail-Proxy Server und ein allgemeiner TCP/UDP Proxy-Server.


\subsubsection{Features}
\label{sec:NginxFeatures}
Die von nginx unterstützten Features werden im Folgenden gegliedert nach den jeweiligen Einsatzgebieten vorgestellt.\cite{.14.03.2017}

\subsubsection{Basic HTTP}
\label{sec:NginxBasicHTTP}
–	Bereitstellung von statischen Dateien
–	Modulare Architektur
–	SSL und TLS SNI support
–	HTTP/2 Unterstützung
–	Rewrite Modul: Änderung von URIs anhand von regulären Ausdrücken
–	Zugriffskontrolle anhand von IP-Adressen oder Passwörtern
–	IP-basierte Geolocation

\subsubsection{Mail Proxy-Server}
\label{sec:NginxMail Proxy-Server}
–	Weiterleitung zu IMAP oder POP3 anhand eines externen HTTP Authentifizierungsserver
–	Authentifizierung durch externen HTTP Authentifizierungsserver und Weiterleitung zu einem internen SMTP-Server
–	SSL Unterstützung
–	STARTTLS und STLS Unterstützung

\subsubsection{TCP/UDP Proxy-Server}
\label{sec:NginxTCP/UDP Proxy-Server}
–	Allgemeines Proxying von TCP und UDP
–	SSL und TLS SNI Unterstützung für TCP
–	Begrenzung von simultanen Verbindungen, die von derselben Adresse kommen
–	Zugriffskontrolle anhand der Benutzeradresse
–	Lastausgleich und Fehlertoleranz

\subsubsection{Architektur und Skalierbarkeit}
\label{sec:NginxArchitektur und Skalierbarkeit}
–	Ein Master und mehrere Worker-Prozesse; Worker brauchen keine Berechtigungen
–	Flexible Konfiguration
