Nginx bestitzt einen Marktanteil von ca. 20 Prozent aller Webserver und ist somit auf Rang 2 der Webserver für aktive Webseiten hinter Apache und vor Microsoft und Google (vgl. \cite{.k}). In dieser Studienarbeit haben wir uns für Nginx entschieden, da ein hoher Wert auf eine flexible Konfiguration und schnelle Serverantworten gelegt wird. Folgend werden die Hauptfeatures von Nginx genauer beschrieben.

\subsubsection{Allgemein}
\label{sec:NginxAllgemein}
Nginx ist ein \ac{HTTP} und Reverse-Proxy Server, ein Mail-Proxy Server und ein allgemeiner \ac{TCP}/\ac{UDP} Proxy-Server. Somit hat Nginx eine hohe Nutzungsbreite.

\subsubsection{Features}
\label{sec:NginxFeatures}
Die von Nginx unterstützten Features werden im Folgenden gegliedert nach den jeweiligen Einsatzgebeiten vorgestellt (vgl. \cite{.14.03.2017}). 

\subsubsection{Basic HTTP}
\label{sec:NginxBasicHTTP}
Nginx unterstützt die Bereitstellung von statischen Dateien. Durch die modulare Architektur ist Nginx für viele Zwecke einsetzbar und bleibt dabei übersichtlich. \ac{HTTP}/2 wird unterstützt, was die Übertragung beschleunigt und optimiert. Das Rewrite Modul ermöglicht die Änderungen von URIs anhand von regulären Ausdrücken. Dies trägt auch zur Suchmaschinenoptimierung bei und ermöglicht das Erstellen von lesbaren \ac{URI}s. Die Zugriffskontrolle kann anhand von \ac{IP}-Adressen oder Passwörtern erfolgen. Anhand der \ac{IP}-Adresse können zudem Informationen zu Geolocation gewonnen werden. \ac{SSL} und \ac{TLS} \ac{SNI} wird natürlich auch unterstützt.

\subsubsection{Mail Proxy-Server}
\label{sec:NginxMail Proxy-Server}
Die Authentifizierung kann über einen externen \ac{HTTP} Server durchgeführt werden, welcher dann nach erfolgreicher Authentifizierung zu einem internen SMTP-Server weiterleitet. Somit muss man sich nicht direkt am Mail-Server authentifizieren. Zudem wird \ac{SSL}, STARTTLS und STLS unterstützt.

\subsubsection{TCP/UDP Proxy-Server}
\label{sec:NginxTCP/UDP Proxy-Server}
Allgemeine Proxy-Features von \ac{TCP} und \ac{UDP} sind vorhanden. Des Weiteren wird \ac{SSL} und \ac{TLS} \ac{SNI} auch für \ac{TCP} unterstützt.
Um vor Überlastungen zu schützen, können Verbindungen, die von derselben Adresse kommen, begrenzt werden. Die Zugriffskontrolle geschieht anhand der Benutzeradresse. Es werden zudem auch Features unterstützt, die den Lastausgleich und die Fehlertoleranz verbessern.


\subsubsection{Architektur und Skalierbarkeit}
\label{sec:NginxArchitektur und Skalierbarkeit}
Nginx ist so aufgebaut, dass es einen Master und mehrere Worker-Prozesse gibt. Die Worker brauchen dabei keinerlei spezielle Berechtigungen, so muss nur dem Master entsprechende Berechtigungen gewährt werden. Des Weiteren ist eine flexible Konfiguration möglich. Die Nginx Konfiguration für diese Studienarbeit befindet sich im Anhang unter Abschnitt \ref {sec:NginxKonfiguration}.
