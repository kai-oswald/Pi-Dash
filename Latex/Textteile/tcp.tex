\ac{TCP} ist ein verbindungsorientiertes Protokoll, dass im \ac{OSI}-7-Schichten Modell auf der vierten Ebene (Transport) einzuordnen ist. Im sogenannten \ac{TCP}/\ac{IP} Protokollstapel ist es in der dritten von vier Schichten zu finden. Bei einer Übertragung von Daten über \ac{TCP} übergibt die genutzte Anwendung den Datenstrom an das Protokoll und empfängt ihn auch wieder von dort. Für die Übertragung ist dementsprechend \ac{TCP} zuständig. 
Die Hauptaufgaben des Protokolls sind daher die Aufteilung und die Zusammensetzung der Daten von entsprechend vielen Paketen (Segmentierung), das Management der Verbindung und ein entsprechendes Fehlerhandling, welches das korrekte Empfangen von Paketen überwacht. Das Fehlerhandling nutzt eine positive Bestätigung aller Pakete. Dies bedeutet, dass nur nicht vorhandene Pakete erneut angefragt werden, ansonsten davon ausgegangen wird, dass die Daten beim Empfänger angekommen sind. Diese Technologie sorgt dafür, dass die Daten auf jeden Fall ankommen, sofern die Verbindung nicht gestört wird (vgl. \cite{.c}\cite{.22.11.2016}).

Der Header eines \ac{TCP} Pakets besteht aus 20 Bytes, jedoch kann dieser auch noch erweitert werden, sodass noch einige zusätzliche Bytes in den Header geschrieben werden. Zu den zwingend notwendigen Daten gehört unter anderem der Port auf dem das Paket empfangen wird und der Port über den das Paket gesendet wird. Zudem wird die Nummer im aktuellen Paketstrom benötigt. Zudem wird eine Prüfsumme und Quittierungsnummer mitgegeben, welche zur Kontrolle und Bestätigung genutzt werden (vgl \cite{.c}).
