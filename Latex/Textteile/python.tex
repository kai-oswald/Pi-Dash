Python ist eine objektorientierte Programmiersprache und wurde 1991 in der ersten Version von Guido van Rossum veröffentlicht (vgl. \cite{.05.03.2017}). Zu den technischen Merkmalen gehört unter anderem die Unterstützung von Paketen beziehungsweise Bibliotheken, die die entsprechend benötigten Funktionen laden, sofern sie in das Projekt eingebunden werden. 

Außerdem ist eines der Merkmale die Einrückung. Python nutzt nicht, im Gegensatz zu anderen Programmiersprachen, bestimmte Klammern oder andere Symbole um den Code zu strukturieren, sondern verwendet stattdessen die Einrückung von den entsprechenden Zeilen Code. Das bedeutet, dass zum Beispiel nach einem Methodenkopf keine geschweifte Klammer oder ähnliches verwendet wird, sondern die nächste Zeile wird um vier Leerzeichen eingerückt. Das Methodenende wird dadurch definiert, dass ein anderer Codeabschnitt wieder auf der gleichen Einrückungsstufe wie der Methodenkopf  beginnt (vgl. \cite{.19.08.2013}\cite{.10.03.2017b}). 
Zudem ist Python eine dynamische Programmiersprache und es ist möglich Python als Skriptsprache zu verwenden und Skripte zu schreiben, die mithilfe des Interpreters ausgeführt werden. Durch die Möglichkeit Python auch auf Linux zu verwenden, können diese Skripte auch entsprechend unter Linux verwendet werden.

Python kann für verschiedene Anwendungszwecke genutzt werden, es gibt zum Beispiel verschiedene Möglichkeiten der Stringverarbeitung, die Möglichkeit mit verschiedenen Internetprotokollen zu arbeiten, beispielsweise HTTP, FTP oder SMTP aber auch auf Interfaces des Betriebssystems zuzugreifen, um beispielsweise mit Sockets (z.B. TCP/IP) zu arbeiten. Neben diesen Funktionen, die durch die Standardbibliotheken abgedeckt werden, gibt es auch noch weitere Pakete, die weitere Bibliotheken mit Funktionen hinzufügen. So gibt es zum Beispiel die Möglichkeit das Paket BeautifulSoup oder Requests hinzuzufügen. Letzteres bietet viele Möglichkeiten mit dem HTTP Protkoll zu arbeiten (vgl \cite{.m}\cite{.l}\cite{.05.03.2017}). 

