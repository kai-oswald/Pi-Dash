\ac{UDP} steht für ``User-Datagram-Protocol'' und ist ein verbindungsloses Transportprotokoll. Im \ac{OSI}-7-Schichten Modell arbeitet es auf der Transportebene und ist für die Zustellung von Netzwerkpaketen von einem Sender zu einem Empfänger zuständig. Im Vergleich zum Transportprotokoll \ac{TCP}, welches verbindungsorientiert arbeitet, ist es wesentlich einfacher zu verarbeiten, da beispielsweise der Header bei den einzelnen Datenpaketen wesentlich kleiner ist. Allgemein ist es sehr minimal gehalten und dadurch sehr einfach zu implementieren und für sehr einfache Anwendungszwecke geeignet. Allerdings ist auch zu erwähnen, dass es keine Empfangsbestätigung gibt und die Daten nach dem Absenden nicht weiter kontrolliert werden. Somit können die Daten auch im Netzwerk verloren gehen und es wird nicht bemerkt. \\
Der einfache Header des \ac{UDP} Protokolls besteht nur aus vier Attributen. Diese jeweils 16 Bit großen Felder enthalten den Quellport, den Zielport, die Checksumme zur Überprüfung des Inhalts und die Länge des gesamten Pakets. Insgesamt ist der Header somit 8 Byte groß. (vgl. \cite{ElektronikKompendium.}\cite{.}\cite{.23.02.2016})
 