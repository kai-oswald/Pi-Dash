Um eine dynamische und nutzerfreundliche Oberfläche zu schaffen wurden Frameworks benutzt. 
Für die Gestaltung der einzelnen Komponenten wurde Bootstrap verwendet, welches für alle wichtigen Komponenten bereits vorgefertigte Styles enthält.
Dies beschleunigt den Prozess der Gestaltung der Oberfläche enorm, da nicht jede verwendete Komponente manuell anhand von \ac{CSS} gestaltet werden muss. 
Die verwendete Bootstrap Version ist Version 3.

Bootstrap ermöglicht es ohne großen Aufwand responsive Webseiten zu gestalten. 
Dabei soll während der Entwicklung auf das mobile-first Prinzip Wert gelegt werden. 
Dieses Prinzip besagt, dass zuerst die Seite für mobile Endgeräte optimiert wird und anschließend erst auf größere Monitore beziehungsweise Auflösungen geachtet werden soll. 
Weltweit liegt der Anteil aller Website-Aufrufe von mobilen Endgeräten bei fast 50\% (vgl. \cite{.stat-mobile}). 

Für die Dynamik sorgt das JavaScript-Framework Vue. Mit Vue vergleichbare Frameworks sind React und Angular. Vue wird verwendet, da es eine große Community hat, eine gute Dokumentation vorweist, sowie eines der besten Frameworks performancetechnisch ist (vgl. \cite{.vue-react-angular}).
Frontend Frameworks, wie auch Vue, bieten die Funktionalität Komponenten zu erstellen und diese beliebig zu verwenden. 
Des Weiteren erhöhen sie, vorausgesetzt sie werden richtig eingesetzt, die Nutzererfahrung, da nicht bei jeder Aktion die komplette Seite neu geladen werden muss.

Im Folgenden wird etwas genauer auf das Vue-Framework eingegangen:
Komponenten sind das Herzstück von Vue. Sie besitzen jeweils genau ein Template, welches in \ac{HTML} definiert ist. 
Des Weiteren haben Komponenten Daten, die durch data-binding automatisiert geändert werden. 
Beispielsweise kann man eine Variable an ein Input-Element binden und wenn dieses einen anderen Wert annimmt wird auch automatisch die Variable angepasst. 
Daten können dabei direkt bei der Initialisierung der Komponente oder zu einem späteren Zeitpunkt geladen werden.
Dadurch wird die initiale Ladezeit der Seite deutlich verringert, da die Daten asynchron geladen werden.
