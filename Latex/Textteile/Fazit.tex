%-> eher auf die Einleitung eingehen 
Im letzten Kapitel dieser Arbeit möchten wir ein abschließendes Fazit ziehen und einen Ausblick geben. Im Rahmen des letzten Kapitels (vgl. \ref{sec:Ergebnis des Projektes-1}) wurde aufgezeigt, dass die Aufgabenstellung dieser Arbeit erfolgreich abgeschlossen werden konnte und die Ergebnisse eine Machbarkeit dieses Projektes bekräftigen. Diesen Punkt möchten wir in unserem Fazit erneut aufgreifen und dabei eine Einschränkung feststellen. Neben der Machbarkeit muss auch auf die Benutzbarkeit dieser Lösung eingegangen werden. An diesem Punkt soll  darauf hingewiesen werden, dass die selbstgebauten Buttons zwar eine Machbarkeit beweisen, aber unter dem Aspekt der Benutzbarkeit die Buttons noch einmal verändert werden sollten. 

Im Rahmen des Projektes ist festzuhalten, dass es nicht darum ging, dass ein möglichst ästhetisch schöner Button erstellt werden soll, der den Herausforderungen einer produktiven Umgebung gewachsen ist. Diese Herausforderungen wären beispielsweise eine erhöhte Luftfeuchtigkeit oder kleinere Kinder, die im Umgang mit elektronischen Teilen unerfahren sind. Im Vergleich zum Amazon Dash Button zeigen die selbstgebauten Dash Buttons einige Mängel auf, allerdings sind wir der Meinung, dass diese durch kleinere Veränderungen, wie zum Beispiel ein entsprechendes Gehäuse auch behoben werden können. 

Allerdings gilt es auch zu beachten, dass diese Buttons aufgrund der Funktionalität durchaus genutzt werden können und somit eine offene und modifizierbare Lösung für (erfahrene) Nutzer bereitstellen können. Auch aufgrund der niedrigen Preise einiger Komponenten (vgl. \cite{.t}) ist es durchaus realistisch, dass diese Buttons auch von Nutzern umgesetzt werden können ohne das umfangreiche finanzielle Investitionen durchgeführt werden müssen. 
Mithilfe dieser Lösung enthält der Nutzer eine offene Lösung, die er erweitern kann. Dies ist insbesondere in Bezug auf das Thema \ac{IoT}, welches bereits in der Einleitung erwähnt wurde (vgl. Kapitel \ref{sec:Einleitung-1}), interessant, da dieses Thema vermutlich in der Zukunft relevanter wird und der Nutzer so seine Lösung mit anderen Lösungen verbinden kann. In Bezug auf das Thema \ac{IoT} und die Buttons ist es den einzelnen Nutzern überlassen, inwiefern er diese Buttonlösung nutzen möchte. Allerdings ist davon auszugehen, dass im Bereich \ac{IoT} weiterhin Produkte auf den Markt kommen, die vermutlich auch Ähnlichkeiten zur Amazon Dash Button Lösung oder der im Rahmen dieser Arbeit erarbeiteten Lösung aufweisen. 

Bei diesen Lösungen sollte der Nutzer die unterschiedlichen technischen Features und Protokolle bzw. Standards genauer betrachten und differenzieren, um eine Lösung zu finden, die seinen Ansprüchen genügt. 
Auch aus diesem Grund wurde eine Lösung entwickelt, die eine Dokumentation aufweist und unterschiedliche Modifikationsmöglichkeiten bietet. So kann ein interessierter Nutzer diese Informationen nutzen und eine eigene Lösung entwickelt, die beispielsweise in den Bereichen Sicherheit, Benutzbarkeit oder auch Mobilität Verbesserungen enthält und somit seinen Ansprüchen besser genügt. 