Wie bereits in der \nameref{sec:Einleitung-1} erwähnt, soll sich diese Studienarbeit mit der Untersuchung des Amazon Dash Buttons beschäftigen und eine Alternative entwickeln. Da bereits während der Recherche vor Projektbeginn herausgefunden werden konnte, dass der Button in der Standardkonfiguration nur mit den Services von Amazon zusammenarbeiten kann, soll eine offenere Lösung entwickelt werden. Mit einer offeneren Lösung wird im Rahmen dieser Studienarbeit eine Möglichkeit definiert, die dem Kunden einen ähnlichen Funktionsumfang anbietet. Ein grundlegender Unterschied ist jedoch im Bestellprozess vorhanden. Während die Lösung von Amazon eine direkte Bestellung bei Amazon auslöst, soll die offenere Lösung eine Art Einkaufszettel bereithalten. 

Der Nutzer soll dann über ein Webfrontend die Möglichkeit haben den Einkaufszettel zu betrachten und dann die entsprechenden Produkte bei seinem bevorzugten Händler zu bestellen. So kann es auch möglich sein, dass auch Produkte geführt werden, die der Nutzer nicht online bestellen kann. Der Vorteil bei dieser Lösung liegt darin, dass der Nutzer nicht an einen Händler und dessen Preise gebunden ist, sondern die Preise betrachten kann und seine Bestellung dementsprechend aufgeben kann. 
Im Rahmen dieses Ziel müssen verschiedene Komponenten entwickelt werden, die im Rahmen der Lösung zusammenarbeiten. Diese Komponenten lassen sich unter folgenden Kategorien zusammenfassen:
\begin{itemize}
\item Entwicklung von Buttons 
\item Entwicklung eines Frontends zur Nutzerinteraktion
\item Entwicklung eines Backends zur Verarbeitung von Frontendeingaben und der Eingabe von Buttons 
\end{itemize}
Die Entwicklung von Buttons lässt sich dabei in zwei Teile aufteilen. Zum einem soll der Amazon Dash Button betrachtet und analysiert werden. In diesem Rahmen soll sowohl die Einrichtung und Konfiguration als auch die Kommunikation untersucht werden. In Folge dessen soll weiterhin geprüft werden, ob der Button auch für die offene Lösung genutzt werden kann und dabei kein Produkt bei Amazon bestellt. Weiterhin sollen mithilfe von Mikrokontrollern eigene Buttons entwickelt werden, die ebenfalls ein Signal an den entsprechenden Empfänger absenden können. Bei den Buttons steht die Funktionalität im Vordergrund und die Prüfung der Machbarkeit des Projektes.


Weiterhin muss ein Frontend entwickelt werden, welches die Nutzerinteraktion ermöglicht. Das Frontend soll zur Übersicht der Einkaufsliste genutzt werden, aber auch die Verwaltung der Buttons und andere anfallende Verwaltungsfunktionen sollen ermöglicht werden. 
Um eine Kommunikation zwischen Frontend und den Buttons zu gewährleisten, muss ein Backend entwickelt werden, welches über Schnittstellen die Kommunikation gewährleistet und die Eingaben entsprechend verarbeitet. 
Als Ergebnis des Projektes soll ein eigener Button entstanden sein, der mithilfe des Backends und Frontends dem Nutzer ermöglicht diesen Button auch zu nutzen. Weiterhin soll zumindest der Amazon Dash Button untersucht worden sein und wenn möglich miteingebunden. Abschließend soll ein Fazit gezogen werden, inwiefern das Projekt umsetzbar ist. 