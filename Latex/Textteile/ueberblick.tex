Die vorgestellte Lösung soll auf freie Technologien zurückgreifen und für den Nutzer nachvollziehbar sein. Das erklärte Ziel ist es, über Buttons einen Einkaufszettel zusammenzustellen, der dem Nutzer dann zur Verfügung gestellt wird. Dazu wird mit sogenannten Mikrocontrollern und Einplatinencomputern die benötigte Infrastruktur realisiert und aufgebaut. Damit die Daten nicht außerhalb der Kontrolle des Nutzers liegen, wird ein zentraler Server im Netzwerk benötigt. Dieser zentrale Server empfängt zudem die entsprechenden Signale der Clients, die in diesem Falle die Buttons sind. 

Ziel ist es, dass eine vergleichbare Lösung aufgebaut wird, die allerdings unabhängig von einem Händler ist und somit an die Bedürfnisse einer Person angepasst werden kann. Damit die Lösung von möglichst vielen Personen genutzt werden kann, soll durchaus auf die Nutzerfreundlichkeit geachtet werden. Allerdings ist zu beachten, dass auch verschiedene Technologien ausgetestet werden, damit man möglichst unterschiedliche Lösungswege betrachten kann. Ein Beispiel , welches in späteren Kapiteln genauer betrachtet wird, wären die unterschiedlichen Übertragungsprotokolle, die bei den Buttons genutzt werden.