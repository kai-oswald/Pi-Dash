Die Bezeichnung Arduino steht für eine Technologie, die sowohl aus Hardware als auch aus Software besteht. Zudem gibt es zwei Unternehmen, die in ihrem Namen den Begriff Arduino tragen. Zum einem gibt es die Arduino LLC, die die Gruppe der Gründer der Plattform bezeichnet. Zudem gibt es die Arduino S.r.l., was die Firma bezeichnet, die anfangs allein die Arduinoboards produzierte und dann auch verkaufte. Die Arduinoplattform wurde ursprünglich entwickelt, um Beginnern den Einstieg in die Mikrokontrollerprogrammierung zu vereinfachen. 
Grundsätzlich besteht die sogenannte Arduinoplattform alber sowohl aus der Hardware als auch der Software. Die Hardware umfasst mittlerweile verschiedene Mikrokontroller beziehungsweise Boards, welche für verschiedene Anwendungszwecke genutzt werden können. Die entsprechende Nutzung wird mithilfe der richtigen Programmierung erreicht. Dafür kann der Softwareteil der Lösung genutzt werden, welches aus einer Entwicklungsumgebung (IDE) besteht und das Schreiben von Programmen vereinfacht. Zudem wird über diese IDE auch die Kommunikation mit dem Mikrokontroller realisiert. Diese Entwicklungsumgebung eignet sich für diverse Mikrokontroller, sofern entsprechende Treiber verfügbar sind, können auch Boards genutzt werden, die nicht direkt mit Arduino zusammenhängen. Zudem kann in verschiedenen Programmiersprachen entwickelt werden, beispielsweise C oder C++. (vgl. \cite{.h,.f,.e,.i,.g,online.})
Die komplette Plattform ist open source, auch wenn die Hardware natürlich bezahlt werden muss. 