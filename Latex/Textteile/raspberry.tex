
\subsubsection{Vorstellung des Raspberry PI}        
\label{sec:Vorstellung des Raspberry PI-1} 

Das Raspberry PI ist ein Einplatinencomputer, den es in verschiedenen Ausführungen gibt. Je nach Ausführung variieren die Ausstattungsmerkmale. Zu den Grundsätzlichen Ausstattungsmerkmalen gehören eine \ac{CPU}, unterschiedlich viel \ac{RAM} und eine \ac{iGPU} Einheit. Er wurde von der Raspberry PI Foundation entwickelt und hatte das ursprüngliche Ziel einen günstigen Computer für den Schulunterricht bereitzustellen. Daher ist es auch möglich eine vollständige Linux Distribution als Betriebssystem zu nutzen und es wurde sogar eine speziell angepasste Distribution veröffentlicht, die Raspbian bezeichnet wird. 

Aufgrund der verschiedenen Möglichkeiten wird er aber mittlerweile auch in vielen anderen Anwendungsgebieten genutzt. Insbesondere die neueren Modelle, die mit mehreren USB Ports, \ac{GPIO} Pins, einem Ethernet Port und weiteren Anschlüssen ausgestattet sind, werden auch in verschiedensten Projekten privater Personen genutzt. Ein weiteres Merkmal ist die Größe des Raspberry PI's, die mit maximal 93mm x 63.5mm x 20mm sehr klein ausfällt. 

Zusätzlich gibt es diverses, bereits für den Raspberry PI ausgelegtes, Zubehör, welches weitere Erweiterungsmöglichkeiten bietet. So gibt es Kameras, Gehäuse, kleine Displays und WLAN Sticks, die den Funktionsumfang erweitern. So wurden bereits diverse Projekte vorgestellt, die zeigen, dass die Einsatzmöglichkeiten des Raspberry Pis wesentlich größer sind (vgl. \cite{.28.12.2016} \cite{.28.01.2017}).



\subsubsection{Verwendung im Projekt}        
\label{sec:Verwendung des Raspberry PI-1} 
Der Raspberry PI wird für das Projekt genutzt, um einen zentralen Server bereitzustellen. Aufgrund der technischen Merkmale kann zeitgleich ein Webserver und ein Datenbankserver betrieben werden. Zudem kann über die USB Anschlüsse ein WLAN Stick angeschlossen werden, sodass die Buttons über das WLAN Netzwerk des Raspberry PIs zum zentralen Server kommunizieren können. 

Dazu muss mithilfe eines Skriptes der WLAN Stick von einem Empfänger zu einem Sender bzw. Access Point umfunktioniert werden. Um die dann eingehenden Nachrichten der Buttons empfangen zu können, muss zudem noch ein entsprechendes Skript im Hintergrund laufen, dass die Daten empfängt. Aufgrund von mehreren parallel laufenden Prozessen (Webserver, Datenbankserver, WLAN Skript und Skripte zum Empfangen der Daten) wird einiges an Rechenleistung benötigt, die der Raspberry Pi allerdings aufbringen kann. 

Der Webserver auf dem Raspberry Pi wird dabei sowohl für das Frontend als auch für das Backend benötigt werden. Das Frontend soll dem Nutzer die Möglichkeit geben, die Liste von Waren zu verwalten und die Buttons zu konfigurieren bzw. Hilfestellung zur Einrichtung zu geben. Das Backend des Servers wird unter anderem aus einer \ac{REST} \ac{API} bestehen, die sowohl die Anfragen des Webservers verarbeitet als auch die Anfragen an die Datenbank im allgemeinen. 
Der Datenbankserver im Hintergrund wird die benötigte Datenbank entsprechend verwalten. 

\begin{figure}[!htb]
	\centering
	\includegraphics[scale=0.5]{Raspberry-Pi-2-web.png}
	\caption[RaspberryPi Modell 2 B]{RaspberryPi Modell 2 B+,\\ Quelle: https://www.raspberrypi.org/wp-content/uploads/2016/02/Raspberry-Pi-2-web.jpg}
\end{figure}